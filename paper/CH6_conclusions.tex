%\documentclass[fleqn]{article}
%\usepackage{graphicx} 
%\usepackage{here}    
%\usepackage{amsmath}
%\textwidth=15.0cm
%\textheight=22.0cm
%\topmargin=-1cm
%\oddsidemargin=-0.3cm
%\evensidemargin=-0.3cm
%
%
%%packages
%\usepackage{amsmath}
%\usepackage{cite}
%\usepackage[toc,page]{appendix}
%\usepackage{fancyvrb}
%\usepackage{tikz}
%\usepackage{multicol}
%\usepackage{framed}
%\usepackage{pgfplots}
%\usepackage{fixltx2e}
%\usepackage{subfigure}
%\usepackage{lscape}
%\usepackage{enumitem}
%\usepackage{multirow}
%\usepackage{color}
%\usepackage{xcolor}
%\usepackage{comment}
%\usepackage{changes}
%\renewcommand*\descriptionlabel[1]{\hspace\leftmargin$#1$}
%
%%tikz labraries
%\usetikzlibrary{matrix}
%\usetikzlibrary{decorations.pathreplacing}
%\usetikzlibrary{positioning}
%\usetikzlibrary{calc}
%\usetikzlibrary{shapes,arrows, chains}
%\usetikzlibrary{intersections}
%\usetikzlibrary{decorations.markings}
%\usetikzlibrary{calc,intersections}
%\usetikzlibrary{patterns}
%
%
%\makeatletter
%\def\mathcolor#1#{\@mathcolor{#1}}
%\def\@mathcolor#1#2#3{%
%  \protect\leavevmode
%  \begingroup
%    \color#1{#2}#3%
%  \endgroup
%}
%\makeatother
%
%%##### DEFINE YOUR NAME #########################
%\definechangesauthor[color=blue,name={Arpad Rozsas}]{AR}
%\definechangesauthor[color=red,name={Nadieh Meinen}]{NEM}
%%###########################################
%
%\begin{document}

\section{Conclusion}\label{conclusions}
In this research an assessment procedure is developed which is able to determine the structural reliability of wind-loaded fa\c{c}ade elements in terms of the failure probability. The assessment procedure accounts for the uncertainties in the extreme wind speeds, the extreme pressure coefficients, the factor correcting for terrain roughness, the structural resistance and the uncertainties in the wind-load model as such. For the modeling of the extreme wind speeds and the extreme pressure coefficients location-specific wind speed and pressure coefficient measurements are being used. The assessment procedure accounts explicitly for the effects of wind-directionality. \\
\\
In order to show its potential, the assessment procedure is applied on a case-study. It is found that the structural reliability of wind-loaded fa\c{c}ade elements is strongly dependent on the building orientation, differing in failure probabilities of several orders of magnitude. Additionally it is found that the failure probability is almost fully determined by the governing incident wind-directions only; the contribution of all other incident wind-directions were found to be negligibly small. \\
\\
For the future developments of the assessment procedure it is recommended to further investigate appropriate statistical methods for the modeling of the extreme wind speeds and the extreme pressure coefficients, to additionally account for the sampling uncertainties in the modeling of the extreme wind speeds and the extreme pressure coefficients, to use realistic resistance models for different failure mechanisms and different material types and, finally, to further investigate the probabilistic description of the roughness factor and the model uncertainty factor.








%\end{document}

