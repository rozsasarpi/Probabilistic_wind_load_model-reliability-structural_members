%\documentclass[fleqn]{article}
%\usepackage{graphicx} 
%\usepackage{here}    
%\usepackage{amsmath}
%\textwidth=15.0cm
%\textheight=22.0cm
%\topmargin=-1cm
%\oddsidemargin=-0.3cm
%\evensidemargin=-0.3cm
%
%
%%packages
%\usepackage{amsmath}
%\usepackage{cite}
%\usepackage[toc,page]{appendix}
%\usepackage{fancyvrb}
%\usepackage{tikz}
%\usepackage{multicol}
%\usepackage{framed}
%\usepackage{pgfplots}
%\usepackage{fixltx2e}
%\usepackage{subfigure}
%\usepackage{lscape}
%\usepackage{enumitem}
%\usepackage{multirow}
%\usepackage{color}
%\usepackage{xcolor}
%\usepackage{comment}
%\usepackage{changes}
%\renewcommand*\descriptionlabel[1]{\hspace\leftmargin$#1$}
%
%%tikz labraries
%\usetikzlibrary{matrix}
%\usetikzlibrary{decorations.pathreplacing}
%\usetikzlibrary{positioning}
%\usetikzlibrary{calc}
%\usetikzlibrary{shapes,arrows, chains}
%\usetikzlibrary{intersections}
%\usetikzlibrary{decorations.markings}
%\usetikzlibrary{calc,intersections}
%\usetikzlibrary{patterns}
%
%
%\makeatletter
%\def\mathcolor#1#{\@mathcolor{#1}}
%\def\@mathcolor#1#2#3{%
%  \protect\leavevmode
%  \begingroup
%    \color#1{#2}#3%
%  \endgroup
%}
%\makeatother
%
%%##### DEFINE YOUR NAME #########################
%\definechangesauthor[color=blue,name={Arpad Rozsas}]{AR}
%\definechangesauthor[color=red,name={Nadieh Meinen}]{NEM}
%%###########################################
%
%\begin{document}

\section{Discussion}\label{discussion}
\textit{Discussion of case-study results}\\
The results show that, due to the wind-directionality effects, the  structural reliability of wind-loaded fa\c{c}ade elements is strongly dependent on the building-orientation, resulting in $\beta_{total}$ values between $3.1\leq\beta_{total}\leq4.6$. In case the wind-directionality effects would not have been taken into account, the outcome of the reliability analysis would be independent of the building orientation and furthermore lower. Especially for the non-governing building orientations this would result in too conservative estimations of the structural reliability. Accounting for wind-directionality effects in the assessment of fa\c{c}ade elements may therefore be extremely advantageous for the design. Similar conclusions have been drawn by Davenport (1983) and Simiu and Filliben (1981).  \\
\\
Furthermore it was found that the total failure probability was almost fully determined by the governing incident wind-directions only; the failure probabilities due to all other incident wind-directions were found to be negligibly small. These results provide arguments for the development of a more practical assessment procedure in the future, where the structural reliability of the fa\c{c}ade element is assessed on the basis of the governing incident wind-directions only. \\
\\
\textit{Discussion of assessment procedure}\\
Even though the results of the case study give good insight in the wind-directionality effects, there are still some remarks to be made concerning the assessment procedure as such. \\
\\
Due to the limited duration of both the wind speed and the wind-tunnel measurements,  the number of sample data available for distribution fitting is relatively small. This introduces large statistical uncertainties in the estimation of the model parameters, which has been shown by Simiu et al. (1978) and Rojiani and Wen (1980). For a better estimation of the structural reliability these statistical uncertainties need to be incorporated in the assessment procedure. \\
\\
Furthermore, for both the extreme wind speeds and the extreme pressure coefficients, the Gumbel fit showed discrepancies with the sample data in the higher probability fractiles. Especially in the case of the extreme pressure coefficients the Gumbel distribution results in a too conservative tail-behaviour. This was also remarked by Kasperski (2003).
In order to assess this discrepancy the distribution skewness is compared with the sample skewness. The sample skewness provides information on the tail behaviour of the random variable, as it is very sensitive for extreme deviations from the mean. In case of the extreme pressure coefficients the sample skewness was found to vary between $-1.65 \leq \hat{\alpha}_c \leq 0.16$. In case of the extreme wind speeds the sample skewness was found to vary between $0.44 \leq \hat{\alpha}_v \leq 1.56$. The  distribution skewness of the Gumbel distribution, however, is fixed to 1.14 for maxima and -1.14 for minima. In many cases this fixed distribution skewness lies far away from the observed sample skewness, explaining the poor fit in the tail.
A better agreement between the sample skewness and the distribution skewness could be obtained the application of a distribution function with a non-fixed distribution skewness, for example the three parameter lognormal distribution or the Weibull distribution. A second option could be the application of a distinct statistical method for the modeling of the extremes, for example the peak-over-threshold method.\\
\\
The structural resistance $R$ was modeled as a lognormally distributed stochastic random variable with a coefficient of variation of $\text{V}_R=0.1$. The design-resistance $R_d$ was assumed to correspond to the EN1990 Level I probability of non-exceedance, with associated sensitivity factor $\alpha_R=0.8$. This chosen sensitivity factor lies far away from the observed sensitivity factor  $\alpha_R=0.22$, which is to be explained by the large uncertainties in the wind-loading part.
For the further development of the assessment procedure it is  recommended to use realistic resistance models for different failure mechanisms and different material types. In this way less assumptions need to be made concerning the design-values of the resistance parameters, as these are fixed according to their characteristic values and partial factors. \\
\\
Finally it is recommended to do further research in the probabilistic modeling of the roughness factor $c_r$ and the model uncertainty factor $\chi_{model}$, which are currently modeled on the basis of approximate literature only. 

\section{Conclusion}\label{conclusions}
In this research an assessment procedure is developed which is able to determine the structural reliability of wind-loaded fa\c{c}ade elements in terms of the failure probability. The assessment procedure accounts for the uncertainties in the extreme wind speeds, the extreme pressure coefficients, the factor correcting for terrain roughness, the structural resistance and the uncertainties in the wind-load model as such. For the modeling of the extreme wind speeds and the extreme pressure coefficients location-specific wind speed and pressure coefficient measurements are being used. The assessment procedure accounts explicitly for the effects of wind-directionality. \\
\\
In order to show its potential, the assessment procedure is applied on a case-study. It is found that the structural reliability of wind-loaded fa\c{c}ade elements is strongly dependent on the building orientation, differing in failure probabilities of several orders of magnitude. Additionally it is found that the failure probability is almost fully determined by the governing incident wind-directions only; the contribution of all other incident wind-directions were found to be negligibly small. \\
\\
For the future developments of the assessment procedure it is recommended to further investigate appropriate statistical methods for the modeling of the extreme wind speeds and the extreme pressure coefficients, to additionally account for the sampling uncertainties in the modeling of the extreme wind speeds and the extreme pressure coefficients, to use realistic resistance models for different failure mechanisms and different material types and, finally, to further investigate the probabilistic description of the roughness factor and the model uncertainty factor.


\nocite{*}
\bibliographystyle{plain}
\bibliography{References}



%
%
%
%\end{document}
%
